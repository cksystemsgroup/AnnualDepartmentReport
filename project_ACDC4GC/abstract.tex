{\bf{Memory Management Benchmarking}}

The performance of managed programming languages is essentially determined by the performance of code execution and the performance of memory management. While there are numerous tools for benchmarking and profiling code execution there is surprisingly little support for systematic analyses of memory management. The available memory management benchmarks typically do not cover the diverse application space of modern VMs such as Google's V8, and do not allow microbenchmarking best- and worst-case behavior. Performance deficiencies in their memory management systems may therefore remain undetected during development and only be exposed later during deployment.

In collaboration with Google Inc., we have designed and implemented a tool called ACDC4GC for benchmarking memory management performance of V8. The tool is implemented in JavaScript for V8 with minimal modifications of the VM. ACDC4GC is based on an existing tool developed by us called ACDC written in C for benchmarking explicit heap allocators. ACDC may be configured to emulate explicit single- and multi-threaded memory allocation, sharing, access, and deallocation behavior to expose virtually any relevant allocator performance differences. ACDC mimics periodic memory allocation and deallocation (AC) as well as persistent memory (DC). Memory may be allocated thread-locally and shared among multiple threads to study multicore scalability and even false sharing. Memory may be deallocated by threads other than the allocating threads to study blowup memory fragmentation. Memory may be accessed and deallocated sequentially in allocation order or in tree-like traversals to expose allocator deficiencies in exploiting spatial locality. We have already demonstrated ACDC’s capabilities with seven state-of-the-art allocators for C/C++ in an empirical study that was presented at ISMM 2013.

ACDC4GC implements a mutator that may be configured to mimic the allocation, access, and deallocation behavior of real and artificial JavaScript applications for exposing virtually any relevant memory management performance characteristics. Similar to ACDC, the new tool emulates object amounts, sizes, liveness (last access), and lifetimes (unreachability) according to configurable distributions that are typically found in JavaScript applications. Unlike ACDC, the new tool also emulates configurable object types beyond list and tree structures to construct realistic and artificial object graphs for benchmarking the performance of reachability analyses. ACDC4GC has also been enhanced to distinguish object liveness and lifetimes according to configurable distributions (rather than a global parameter as in ACDC) for emulating application deficiencies including reachable memory leaks. ACDC4GC was presented at DLS 2014.