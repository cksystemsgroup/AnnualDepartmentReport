{\bf{Cyber-Physical Cloud Computing (CPCC)}}

How do we program, control, and utilize large networks of mobile devices from cell phones to autonomous vehicles? Our idea is to virtualize such systems and create cyber-physical clouds of virtual vehicles that may be rented and operated by customers just like traditional clouds of virtual machines. The key difference here is that virtual vehicles not only compute but also sense, act, and move in space. Cyber-Physical Cloud Computing promises similar benefits as traditional cloud computing by turning mobile devices into a utility. Customers may focus on the tasks they are interested in and only pay for them while providers may focus on maintaining and operating hardware. The idea opens up a whole new space of interesting problems from spatial scheduling and queueing theory to the engineering of virtual vehicles.

The project is a collaborative effort between the University of California at Berkeley and the University of Salzburg. It is funded by the National Science Foundation in the US. A number of students from Salzburg have already visited UC Berkeley and worked on the project in Berkeley.